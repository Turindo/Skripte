\documentclass[10pt,a4paper,titlepage]{book}
\usepackage[utf8]{inputenc}
\usepackage[german]{babel}
\usepackage[T1]{fontenc}
\usepackage{amsmath}
\usepackage{amsfonts}
\usepackage{amssymb}
\title{Einführung in die Wahrscheinlichkeitstheortie}
\usepackage[left=2cm,right=2cm,top=2cm,bottom=2cm]{geometry}
\begin{document}
\maketitle
\newpage
\tableofcontents
\newpage

Literatur: Sheldon Ross: Introduction to probability models.
\section{0. Einführendes Beispiel}
\paragraph{Münzexperiment}
Bei 50 aufeinanderfolgenden Würfen einer fairen Münze. Mit welcher Wahrscheinlichkeit erscheint im laufe der Würfe 5 mal hintereinander Zahl?

Antwort: die WSK beträgt ca. 0,55.

Dieses Bsp. zeigt:
\begin{itemize}
\item intuitive Schätzung ist oft weit von der tatsächlichen WSK entfernt.
\item Pechsträhne bei Münzwürfen sehr häufig
\end{itemize}
\section{1. Modelierung von Zufallsexperimenten}
\subsection{Ergebnisräume und Ereignisse}
Ein Ergebnisraum (ER) ist eine Menge, die alle möglichen Ausgänge eines Zufallsexperiments umfasst.
Bezeichnung: $\Omega$.

Beispiele für Ereignisräume:
\begin{itemize}
\item [a)] Zufallsexperiment ist ein einmaliger Münzwurf:
$$\Omega = \{K, Z\}$$
\item [b)] bei zweifachem Münzwurf:\\
$$\Omega = \{K, Z\}^2 = \{(K,K), (K,Z), (Z,K), (Z,Z)\}$$
"kartesisches Produkt"
\item [c)] einfacher Münzwurf:
$$\Omega = \{1,2,3,4,5,6\}$$
\item [d)] zweifacher Münzwurf:
$$\Omega = \{1,2,3,4,5,6\}^2$$
\item [e)] erzielte Tore im einem Fußballspiel:
$$\Omega =  \mathbb{N}_{\geq 0}:=\{0,1,2,...\}$$
\end{itemize}

Vorerst: nur ER mit abzählbar vielen Elementen

Wir bezeichnen jede Teilmenge von $\Omega$ als ein Ereignis. Man sagt: Ein Ereignis $A \subset \Omega$ tritt ein, falls das Ergebnis des Zufallsexp. in $A$ liegt.

Beispiele für Ereignisse:
\begin{itemize}
\item [a)] Sei $\Omega = \{K,Z\}$ (zweifacher Münzwurf)\\

\begin{tabular}{l|l}
Ereignis in Worten          & Ereignis als Teilmenge  \\ \hline
1. Wurf ist Zahl            & \{(Z,K), (Z,Z)\}        \\
Höchstens ein Wurf ist Zahl & \{(Z,K), (K,K), (K,Z)\}
\end{tabular}

\item [b)] Sei $\Omega = \mathbb{Z}_{\geq 0}$ (Tore im Fußball)\\

\begin{tabular}{l|l}
Ereignis in Worten          & Ereignis als Teilmenge  \\ \hline
höchstens drei Tore           & \{0,1,2,3\}        \\
mindestens ein Tor & $\mathbb{Z}_{> 0}$ \\
gerade Anzahl an Toren & \{0,2,4,6,...\}
\end{tabular}
\end{itemize}

Seien A,B zwei Ereignosse von $\Omega$.
$$A \subset \Omega$$
$$B \subset \Omega$$

Neue Ereignisse:
TABELLE

Ein Ereignis heißt Elementarereignis oder Ergebnis, falls das Ereignis nur ein Element enthält. Wir bezeichnen mit $\mathcal{P}(\Omega) := \{A: A\subset \Omega\}$ die Potenzmenge von $\Omega$. $\mathcal{P}(\Omega)$ ist häufig sehr viel größer als der Ergebnisraum $\Omega$.

\subsection*{1.2 Wahrscheinlichkeitsmaß (WM)}
\newcommand{\abbildung}{$P: \mathcal{P}(\Omega) \rightarrow [0;1]$}
\paragraph{Definition 1.1}  Sei $\Omega \neq \emptyset$ abzählbar. Das WSK-maß ist eine Abbildung $P:\mathcal{P}(\Omega)$.
$$P: \mathcal{P}(\Omega) \rightarrow [0;1]$$
mit zwei Eigenschaften.
Eine Abbildung \\abbildung heißt WM, falls gilt:
\begin{itemize}
\item[(W1)] $P(\Omega) = 1$
\item[(W2)] Sind $A_1, A_2, ..$ disjunkte Ereignisse (d. h. $A_i \cap A_j = \emptyset$, falls $i \neq j$)
$$P\left(\bigcup_{n=1}^\infty\right)  = \sum_{n =1}^\infty P(A_n)$$ HIER FEHLT ETWAS
\end{itemize}

Man sagt $P(A)$ ist die WSK des Ereignisses A.

Beispiele für WSK-maße
\begin{itemize}
\item[a)]  Modell des einmaligen Münzwurfes einer fairen Münze
$\Omega = \{K,Z\}$ und P WM mit $P(\{K\}) = P(\{Z\}) = \frac{1}{2}$ 
\item[b)] Modell eines fairen Würfelwurfs $\Omega = {1,2,3,4,5,6}$ und P WM mit $P({1}) = P({2}) = ...  = P({6}) = \frac{1}{6}$
\item[c)] La-Place-Experiment: alle Ergebnisse mind. Gleichwahrscheinlich (Verallgemeinerung von a) und b)). Sei $\Omega$ eine endliche Menge und P WM $$P(A) := \frac{|A|}{|\Omega|}, A \in \mathcal{P}(\Omega)$$

$|A|$-Möglichkeit von A-Kardinalität: Anzahl der Elemente in A
\end{itemize}

\paragraph{Lemma / Satz 1.2 (Eigenschaften von WM)}

Sei \abbildung ein WM. Dann gilt:
\begin{itemize}
\item[(1)]$P(\emptyset) = 0$
\item[(2)]$A_1, ..., A_n$ disjunkt $\Rightarrow P\left(\bigcup_{i=1}^n A_i\right) = \sum_{i=1}^n P(A_i)$ (endliche Additivität)
\end{itemize}
Des Weiteren gilt für alle $A, B \in \mathcal{P}(\Omega)$.

\begin{itemize}
\item[(3)]$A \subset B \Rightarrow P(A) \le P(B)$ (Monotonieeigenschaft des WM)
\item[(4)] $P(A^C) = 1-P(A)$
\item[(5)] $P(A \setminus B) = P(A) - P(A \cap B)$
\item[(6)] $P(A \cup B) = P(A) + P(B) - P(A \cap B)$
\end{itemize}

\paragraph{Beweis}
\begin{itemize}
\item[(1)]Setzen $A_n := \emptyset$ für alle $n \geq 1$\\
Dann folgt aus (W2)
$$P(\emptyset) = P\left(\bigcup_{k=1}^\infty \right) = \sum_{n=1}^\infty P(A_n) = P(\emptyset) + P(\emptyset) + ...$$
Da $P(\emptyset) \in [0;1]$ muss $P(\emptyset) = 0$ gelten.
\item[(2)] Sei $A_1, ..., A_n$ beliebige disjunkte Ereignbisse und $A_m \neq \emptyset$ für alle $m > n$. Dann folgt aus (W2) und (1)
$$P\left(\bigcup_{i=1}^n  A_i\right) = \underbrace{ P\left(\bigcup_{i=1}^\infty  A_i\right)}_{\text{Bei } i \geq (n+1) \text{ füge } \emptyset \text{ hinzu}} = P(A_i) = \sum_{i=1}^\infty P(A_i) = \sum_{i=1}^n P(A_i)$$

\item[(3)] Sei $A \subset B$. Dann ist disjunkte Vereinigung von $A$ und $B \setminus A$. 
\\Wegen (2) gilt $P(B) = P(A) + \underbrace{P(B \setminus A)}_{\geq 0} \geq P(A)$

\item[(4)] Beachte: $\Omega$ ist disjunkte Vereinigung von $A$ und $A^C$.\\
Also wegen (2):
$P(\Omega) = P(A) + P(A^C)$\\
Da $P(\Omega) = 1$, folgt $P(A^C) = 1-P(A)$

\item[(4)] $P(A \cup B) = P(A) + P(B) - P(A \cap B)$
\item[(5 \& 6)] ÜBUNG

\end{itemize}

\fbox{\begin{minipage}{0.8\textwidth}
\paragraph{Bemerkung:} Sei $\Omega$ abzählung und P ein WM auf $\mathcal{P}(\Omega$). Jedes Ereignis $A \subset \Omega$ ist abzählbare Vereinigung der Elementarereignisse $\{\omega\}$ mit $\omega \in A$. Also folgt aus (W2) und (2):
\begin{align}
P(A) = \sum_{\omega \in A} P(\{\omega\})
\end{align}
\end{minipage}}

Ausblick: Man kann WM auf überabzählbaren ER definieren: Eigenschaften (1)-(6) gelten dann noch, aber Eigenschaft (1.1.) nicht mehr!

\subsubsection{\underline{Beispiel 1.3} (Lange Sequenzen von Zahl-Würfen)}

Ziel Methode, um Frage aus Kap. 0. zu beantworten.

Dazu: Wie groß ist die WSk, dass bei n Würfen einer fairen Münze höchstens $x$ mal hintereinander $Z$ (= Zahl) erscheint?

$\rightarrow$ müssen dafür ein Modell erstellen:
\begin{itemize}
\item Erhgebisraum
\item WSK-Maß
\end{itemize}
Hierzu sei \\$\Omega = \{K,Z\}^n$ \footnote{endlich, damit abzählbar $\rightarrow$ (1.1) anwendbar} und $P_n$ WM mit $P_n(\{\omega\}) = \frac{1}{|\Omega_n|} = \frac{1}{2^n}$
\\
$A_n =$  "Keine Z-Teilsequenz ist länger als x"\footnote{Wenn die Ereignisse in Worten knackig ausgedrückt werden können, kann und sollte man eine solche Umschreibung angeben.}\\
$a_n(x) := |A_n(X)|$\\
Beachte: falls $x>=n$, dann ist $a_n(x)=2^n$\\
Für später setzen wir $a_0(x)=1$\footnote{Rekursionsanfang}. Für festes x lässt sich $a_n(x)$ rekursiv in n berechnen.\\

Beispiel: jede Folge in $A_n(2)$ startet mit K,ZK oder ZZK.
Beobachte: In $A_n(2)$ gibt es
\begin{itemize}
\item $a_n-1(2)$ Elemente, die mit K starten
\item $a_n-2(2)$ Elemente, die mit ZK starten
\item $a_n-3(2)$ Elemente, die mit ZZK starten
\end{itemize}

Also  gilt für alle $n>=3$:
$$a_n(2) = a_{n-1}(2)+a_{n-2}(2)+a_{n-3}(2)$$
Allgemein lässt sich zeigen:
\begin{align}
x)=\sum_{j=0}^x a_{n-j-1}(x) \text{ für alle } n >= x+1
\end{align}

Die gesuchte WSK ist
\begin{align}
P_n(A_(x)) = \frac{a_n(x)}{2^n}
\end{align}

Die Formeln (1.2) und (1.3) lassen sich gut implementieren\footnote{vgl. Übungsblatt 2}

Im Bsp. aus Kap. O ist die WSK des \underline{Komplementes von $A_{50}(4)$} gesucht.
Numerische Berechnung: $$P_50(A_{50}(4)^C) = 1-P((A_{50}(4)) = 1-\frac{a_{50}(4)}{2^50} \approx 0,55$$

\subsection{Exkurs: Nützliche Formeln der Kombinatorik}
Ziel: Formeln, die hilfreiche sind zu Bestimmung von Mächtigkeiten.
Zur Illustration denken wir an eine Urne mit $n$ gleichförmigen Kugeln, die mit $1$ bis $n$ durchnummeriert sind.

\begin{enumerate}
\item \underline{Variationen} (alt. geordnete Stichproben) \underline{mit Widerholung}\\ 
Man zieht k mal mit Zurücklegen; Reihenfolge ist wichtig.\\
Die Anzahl möglicher Ergebnisse: \textbf{$n^k$}
\item \underline{Variationen ohne Wiederholung}\\
Man zieht k mal ohne Zurücklegen; Reihenfolge ist wichtig.

Die Anzahl möglicher Ergebnisse:
$$n*(n+1)*...*(n-k+1)) = \frac{n!}{(n-k)!}$$

\item Kombinationen (alt. ungeordnete Stichproben) ohne Widerholung\\
Man zieht k mal ohne Zurücklegen, Reigenfolge ist egal.
Die Anzahl möglicher Ergebnisse: !!!!!!!!!!!!!!!! (n über k)

\item Kombinationen mit Wiederholung
Man zieht k mal mit Zurücklegen, Reihenfolge ist egal.
Die Anzahl möglicher Ergebnisse:

Begründung anhand eines Beispiels:
Sei $\underbrace{n=5}_{\text{Farben}}$ und $\underbrace{k = 3}_{\text{Züge}}$.
Jede Kombination mit Wdh. entspricht einer Folge mit Symbolen | und *.
\end{enumerate}

HIER FEHLT NE MENGE

Anwendungsbeispiele in der W-Theorie

\begin{enumerate}
\item In der Übung sitzen 23 Studierende. Mit welcher WSK haben alle an unterschiedlichen Tagen Geburtstag= (Ohne Schaltjahr)
$n = 365$; $k=23$; mit Wiederholung\\
1. Fall): Variationen
Modell: $\Omega$ = Menge der Variationen mit Wdh., wobei $n =365$ und $k=23$. P ist WM mit $P(\{\omega\})=\frac{1}{|\Omega|}$ (Laplace)\\
Beachte: $|\Omega| = 365^{23}$ $(\text{a)})$
Das Ereignis "alle an unterschiedlichen Tagen Geburtstag" ist $A$ = Menge der Variationen ohne Wdh.
Die gesuchte WSK ist $$P(A)=\frac{|A|}{|\Omega|}=\underbrace{\frac{365!}{(365-23)!}}_{|A|}*\underbrace{\frac{1}{365^{23}}}_{|\Omega|}\approx 0,49$$\\
Fazit: Es ist wahrscheinlicher, dass 2 Studierende am selben Tag Geburtstag haben als, dass dies nicht der Fall ist!
\item Man zieht 2 mal ohne Zurücklegen aus einer Urne mit 7 Schwarzen und 5 weißen Kugeln. Mit welcher WSK zieht man 2 mal schwarz?
1. Trick: nichts hindert uns daran, uns die Kugeln nummeriert zu denken
Für die Beantwortung denkt man sic die Kuggeln durchnummeriert. Die ersten 7 Kugeln sind schwarz, die letzten 5 weiß.

Modell:
$\Omega = $ Menge der Kombinationen ohne Wiederholung mit n = 12, k = 2. P WM mit $P(\{\omega\} = \frac{1}{|\Omega|})$.
Beachte: $|\Omega| = ... = \frac{12!}{(12-2! 2!)} = \underline{66}$
Das Ereignis "Beide Schwarz” ist $A=\{\{\omega_1, \omega_2\}: \omega_1, \omega
_2 \in \{1, ..., 7\} \text{ und } \omega_1 \neq \omega_2\}$

Beachte: $|A| = ()$
\end{enumerate}



\end{document}
