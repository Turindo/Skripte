\documentclass[10pt,a4paper,titlepage]{book}
\usepackage[utf8]{inputenc}
\usepackage[german]{babel}
\usepackage[T1]{fontenc}
\usepackage{amsmath}
\usepackage{amsfonts}
\usepackage{amssymb}
\title{Einführung in die Wahrscheinlichkeitstheortie}
\usepackage[left=2cm,right=2cm,top=2cm,bottom=2cm]{geometry}
\begin{document}
\maketitle
\newpage

Literatur: Sheldon Ross: Introduction to probability models.
\section*{0. Einführendes Beispiel}
\paragraph{Münzexperiment}
Bei 50 aufeinanderfolgenden Würfen einer fairen Münze. Mit welcher Wahrscheinlichkeit erscheint im laufe der Würfe 5 mal hintereinander Zahl?

Antwort: die WSK beträgt ca. 0,55.

Dieses Bsp. zeigt:
\begin{itemize}
\item intuitive Schätzung ist oft weit von der tatsächlichen WSK entfernt.
\item Pechsträhne bei Münzwürfen sehr häufig
\end{itemize}
\section*{1. Modelierung von Zufallsexperimenten}
\subsection*{Ergebnisräume und Ereignisse}
Ein Ergebnisraum (ER) ist eine Menge, die alle möglichen Ausgänge eines Zufallsexperiments umfasst.
Bezeichnung: $\Omega$.

Beispiele für Ereignisräume:
\begin{itemize}
\item [a)] Zufallsexperiment ist ein einmaliger Münzwurf:
$$\Omega = \{K, Z\}$$
\item [b)] bei zweifachem Münzwurf:\\
$$\Omega = \{K, Z\}^2 = \{(K,K), (K,Z), (Z,K), (Z,Z)\}$$
"kartesisches Produkt"
\item [c)] einfacher Münzwurf:
$$\Omega = \{1,2,3,4,5,6\}$$
\item [d)] zweifacher Münzwurf:
$$\Omega = \{1,2,3,4,5,6\}^2$$
\item [e)] erzielte Tore im einem Fußballspiel:
$$\Omega =  \mathbb{N}_{\geq 0}:=\{0,1,2,...\}$$
\end{itemize}

Vorerst: nur ER mit abzählbar vielen Elementen

Wir bezeichnen jede Teilmenge von $\Omega$ als ein Ereignis. Man sagt: Ein Ereignis $A \subset \Omega$ tritt ein, falls das Ergebnis des Zufallsexp. in $A$ liegt.

Beispiele für Ereignisse:
\begin{itemize}
\item [a)] Sei $\Omega = \{K,Z\}$ (zweifacher Münzwurf)\\

\begin{tabular}{l|l}
Ereignis in Worten          & Ereignis als Teilmenge  \\ \hline
1. Wurf ist Zahl            & \{(Z,K), (Z,Z)\}        \\
Höchstens ein Wurf ist Zahl & \{(Z,K), (K,K), (K,Z)\}
\end{tabular}

\item [b)] Sei $\Omega = \mathbb{Z}_{\geq 0}$ (Tore im Fußball)\\

\begin{tabular}{l|l}
Ereignis in Worten          & Ereignis als Teilmenge  \\ \hline
höchstens drei Tore           & \{0,1,2,3\}        \\
mindestens ein Tor & $\mathbb{Z}_{> 0}$ \\
gerade Anzahl an Toren & \{0,2,4,6,...\}
\end{tabular}
\end{itemize}

Seien A,B zwei Ereignosse von $\Omega$.
$$A \subset \Omega$$
$$B \subset \Omega$$

Neue Ereignisse:
TABELLE

Ein Ereignis heißt Elementarereignis oder Ergebnis, falls das Ereignis nur ein Element enthält. Wir bezeichnen mit $\mathcal{P}(\Omega) := \{A: A\subset \Omega\}$ die Potenzmenge von $\Omega$. $\mathcal{P}(\Omega)$ ist häufig sehr viel größer als der Ergebnisraum $\Omega$.

\subsection*{1.2 Wahrscheinlichkeitsmaß (WM)}
\newcommand{\abbildung}{$P: \mathcal{P}(\Omega) \rightarrow [0;1]$}
\paragraph{Definition 1.1.}  Sei $\Omega \neq \emptyset$ abzählbar. Das WSK-maß ist eine Abbildung $P:\mathcal{P}(\Omega)$.
$$P: \mathcal{P}(\Omega) \rightarrow [0;1]$$
mit zwei Eigenschaften.
Eine Abbildung \\abbildung heißt WM, falls gilt:
\begin{itemize}
\item[(W1)] $P(\Omega) = 1$
\item[(W2)] Sind $A_1, A_2, ..$ disjunkte Ereignisse (d. h. $A_i \cap A_j = \emptyset$, falls $i \neq j$)
$$P\left(\bigcup_{n=1}^\infty\right)  = \sum_{n =1}^\infty P(A_n)$$ HIER FEHLT ETWAS

Man sagt $P(A)$ ist die WSK des Ereignisses A.

Beispiele für WSK-maße

\end{itemize}


\end{document}
